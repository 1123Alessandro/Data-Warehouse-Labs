\chapter{Write-Up, thoughts, reviews, comparisons}

The Lab Exercise \#2 significantly departed from our previous experience in Lab Exercise \#1. Our group delved into the fascinating world of data modeling and schema design in this exercise. Specifically, we explored the dimensional model, a vital approach for structuring data in data warehousing and business intelligence scenarios.

The central focus of this exercise was the creation of a star schema, a powerful way to organize data into dimensions and facts. Our schema consisted of dimensions such as courses, students, enrollments, and admissions, with the student enrollment fact table at the core. Unlike the normalized schema, which prioritizes eliminating data redundancy and maintaining data integrity, the star schema emphasizes simplifying data for analytical purposes.

As we worked on this exercise, we grew to appreciate the nuances of both schema designs. With its complex web of interconnected tables, the normalized schema taught us the importance of data integrity and consistency. It demonstrated how to eliminate data anomalies and ensure that updates and modifications to the database would not result in inconsistencies.

However, the limitations of the normalized schema became evident when it came to analyzing and reporting on the data. Queries necessitated intricate JOIN operations across multiple tables, making them simple and efficient for business intelligence purposes. These are where the dimensional model schema shone. It introduced the concepts of facts and dimensions, with facts representing quantitative data like tuition fees, while dimensions simplified the schema.

Creating a dimensional model schema was truly eye-opening. We observed how this approach streamlined reporting and data analysis. Queries that were once convoluted with JOINs became straightforward, making extracting meaningful insights from the data more accessible. We realized that dimensional modeling was particularly well-suited for data warehousing and business intelligence scenarios, where quick access to summarized data was crucial.

In conclusion, our experience with Lab Exercise \#2, using both dimensional and normalized schema diagrams, has been invaluable. While the normalized schema emphasized data integrity and consistency, the dimensional model schema showcased the significance of data accessibility and analysis. As future data science professionals, understanding when to use each schema approach is a valuable skill we will carry forward in our academic and professional journey.

\section{Difference of Lab Exercise 1 and Lab Exercise 2}

The querying approaches in Lab Exercise \#1 and Lab Exercise \#2 diverge significantly, reflecting the unique characteristics of each exercise's objectives and database schemas. In Lab Exercise \#1, the primary focus was on SQL querying within a normalized schema. Here, SQL queries were employed to retrieve, filter, and manipulate data from a database structured into multiple related tables. While this normalization strategy was geared towards maintaining data integrity by eliminating redundancy and anomalies, it often entailed intricate JOIN operations to access information across various tables. 

The Lab Exercise \#2, on the other hand, centered on dimensional modeling. It entailed designing a star schema tailored for business intelligence and analytical reporting purposes. This schema simplified data organization by introducing a central fact table (e.g., student enrollment) surrounded by dimension tables (e.g., courses, students). This simplification led to more efficient and intuitive queries, as the need for complex JOINs was reduced. The introduction of fact and dimension tables facilitated data accessibility and analysis, making Lab Exercise \#2 particularly suitable for data warehousing and reporting scenarios where quick access to summarized data is paramount. Thus, while Lab Exercise \#1 underscored the significance of data integrity and complex JOIN operations within a normalized schema, Lab Exercise \#2 highlighted the importance of data accessibility, analysis, and efficient querying within a dimensional model schema, emphasizing their relevance in the broader field of data warehousing and business intelligence.
